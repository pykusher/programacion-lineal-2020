\documentclass{article}
\usepackage[utf8]{inputenc}
\usepackage{amsmath}
\usepackage[spanish]{babel}
\title{Apuntes de programación lineal}
\author{CypressKusher}


\begin{document}



\maketitle
\section{Introducción}
\label{sec:introduccion}

La forma estándar de un problema de progamación lineal es:\\
Dados una matriz $A$ y vectores $b,c$, maximizar $c^{T}x$ sujeto a
$Ax\leq b$.

\begin{tabular}{|c|c|c|}
  \hline
  &A&B\\
  \hline
  Máquina 1&1&2\\
  Máquina 2&1&1\\
  \hline
\end{tabular}

\begin{equation*}
  \label{eq:2}
 A= \begin{pmatrix}
    a&1&2\\
    3&-1&5
    
  \end{pmatrix}
\end{equation*}
\midskip
\begin{equation}
  \label{eq:2}
 A= \begin{pmatrix}
    a&1&2\\
    3&-1&5
  \end{pmatrix}
\end{equation}
Para hacer saltos dentro de ecuacoines se usan los comandos smallskip,
bigskip, midskipe.
Latex divide la palabra según la optimizacion de espacio que creea
mejor, por lo que aveces divide las palabras sin tener mucho sentido
en español. El paquete babael spanish.


\end{document}
